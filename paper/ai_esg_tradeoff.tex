\documentclass[12pt,a4paper]{article}

% Packages
\usepackage[utf8]{inputenc}
\usepackage[T1]{fontenc}
\usepackage{amsmath,amssymb}
\usepackage{graphicx}
\usepackage{booktabs}
\usepackage{natbib}
\usepackage{hyperref}
\usepackage{geometry}
\usepackage{setspace}
\usepackage{float}
\usepackage{caption}
\usepackage{subcaption}

\geometry{margin=1in}
\doublespacing

% Title
\title{AI Adoption and the ESG Trade-off: Evidence from Corporate Emissions Data}
\author{
    Alina Malkova\\
    \textit{Florida Institute of Technology}\\
    \texttt{amalkova@fit.edu}
}
\date{\today}

\begin{document}

\maketitle

\begin{abstract}
Does artificial intelligence adoption create a productivity-ESG trade-off for public firms? Using the ChatGPT launch (November 2022) as an exogenous shock to AI adoption pressure, I employ a difference-in-differences design to examine whether firms in high AI-exposed industries experienced differential changes in greenhouse gas emissions. Matching EPA GHGRP facility-level emissions data to S\&P 500 firms (2010--2023), I find no significant differential effect of AI exposure on Scope 1 emissions---but this null reflects measurement mismatch rather than absence of an effect. AI infrastructure emissions are primarily Scope 2 (purchased electricity), which GHGRP does not capture. Using an expanded panel of manually collected Scope 2 data from corporate sustainability reports (1,423 observations for 319 S\&P 500 firms across 12 GICS sectors, 2016--2024), I find Big Tech Scope 2 grew substantially from 2019--2023, with Meta showing +223\%, Amazon +97\%, Microsoft +77\%, and Alphabet +47\%. A firm-level DiD using 219 companies with complete 2019--2023 panels estimates a statistically significant +73\% AI builder effect post-ChatGPT ($\beta = 0.55$, $p = 0.009$). Utility-level analysis confirms 9.1\% differential electricity demand growth in data center hub states ($p = 0.007$). An instrumental variables strategy using pre-determined data center siting characteristics (tax incentives, IXP proximity, electricity rates) finds that high-suitability states experienced 3.1\% differential Scope 2 emissions growth post-ChatGPT ($p < 0.001$). The contrast---null on Scope 1, positive on Scope 2---reveals that current ESG frameworks systematically understate AI's environmental footprint, as Information Technology shows 84\% Scope 2 share versus Utilities at 17\% and Energy at 14\%.

\medskip
\noindent\textbf{Keywords:} Artificial Intelligence, ESG, Carbon Emissions, Scope 2, Data Centers, ChatGPT

\noindent\textbf{JEL Codes:} G30, Q54, O33, M14
\end{abstract}

\newpage

\section{Introduction}

The rapid adoption of artificial intelligence technologies has created a fundamental tension for corporate environmental sustainability. Firms that aggressively deploy AI---and by extension, the data centers with massive energy footprints that power these systems---may boost productivity and competitiveness but simultaneously increase their carbon emissions. This tension is particularly acute given that environmental, social, and governance (ESG) performance increasingly matters for institutional investors, index inclusion, and cost of capital.

This paper investigates whether AI adoption creates a measurable productivity-ESG trade-off for public firms. I exploit the launch of ChatGPT in November 2022 as an exogenous shock that dramatically increased competitive pressure to adopt AI across industries. Using a difference-in-differences framework, I compare emissions trajectories of firms in high versus low AI-exposed industries before and after this shock.

My analysis reveals a critical measurement challenge: the EPA's Greenhouse Gas Reporting Program (GHGRP)---the primary regulatory source of U.S. corporate emissions data---captures only Scope 1 emissions (direct combustion from owned facilities). However, AI infrastructure emissions are predominantly Scope 2 (indirect emissions from purchased electricity to power data centers). I document that for major technology firms, Scope 2 represents 87--99\% of total emissions, and these emissions grew 60--176\% between 2020 and 2023. None of this growth appears in GHGRP data.

This measurement gap has significant implications for both academic research and policy. Studies relying on regulatory emissions databases may systematically underestimate the environmental costs of AI adoption. Current ESG frameworks may not adequately capture emissions from the fastest-growing source of corporate carbon footprints.

\section{Related Literature}

\subsection{AI Adoption and ESG Performance}

The dominant finding in recent literature is that AI adoption \textit{improves} firm-level ESG performance. Studies using Chinese A-share listed firms find positive effects across all three ESG pillars, with mechanisms including better internal controls, financing constraint alleviation, and green innovation \citep[see][for a review]{chen2025ai}. However, this literature treats AI as a tool for ESG management rather than examining the equilibrium trade-off firms face when AI adoption itself carries environmental costs.

\subsection{Big Tech's Emissions Crisis}

Emerging evidence contradicts the optimistic view for firms building AI infrastructure. Alphabet's emissions rose nearly 50\% since 2019; Meta's location-based emissions more than doubled; Microsoft's rose 23.4\% since 2020---all driven by data center electricity demand. In 2025, Microsoft, Google, Amazon, and Meta are projected to spend a combined \$320 billion on AI infrastructure. A Harvard study found that the carbon intensity of electricity used by data centers was 48\% higher than the U.S. average.

\subsection{Emissions Scope and Measurement}

Corporate emissions are classified into three scopes: Scope 1 (direct emissions from owned sources), Scope 2 (indirect emissions from purchased energy), and Scope 3 (all other indirect emissions in the value chain). For technology firms, Scope 2 dominates because data centers consume massive amounts of electricity but generate minimal direct emissions. The EPA GHGRP requires reporting only for facilities emitting more than 25,000 metric tons CO$_2$e of direct (Scope 1) emissions, creating a systematic gap for electricity-intensive operations.

\section{Data}

\subsection{EPA GHGRP Emissions Panel}

I obtain facility-level emissions data from the EPA Greenhouse Gas Reporting Program for 2010--2023. Matching facilities to parent companies using EPA ownership data and then to S\&P 500 constituents, I construct a panel of 121 firms with 1,636 company-year observations. The panel is well-balanced: 109 firms have complete data for all 14 years.

\subsection{AI Exposure Index}

Following \citet{felten2021occupational}, I construct an AI exposure index using O*NET occupational ability and work activity data. I identify abilities where AI systems have strong capabilities (e.g., written comprehension, deductive reasoning, information processing) and weight occupations by their reliance on these abilities. Aggregating to GICS sectors, I find Information Technology (81.5) and Financials (81.2) have the highest AI exposure, while Utilities (42.0) and Materials (37.2) have the lowest.

\subsection{CDP Scope 2 Data}

To examine the measurement gap, I obtain corporate emissions data from CDP (formerly Carbon Disclosure Project) for 2011--2013, which includes both Scope 1 and Scope 2 emissions.

\subsection{Manual Scope 2 Panel}

To enable direct DiD estimation on Scope 2 emissions, I manually collect emissions data from corporate sustainability reports for 319 S\&P 500 firms across all 12 GICS sectors (1,423 company-year observations, 2016--2024). The expanded sample provides comprehensive coverage including: (1) AI Infrastructure Builders (MSFT, GOOGL, META, AMZN, AAPL, NVDA, ORCL, IBM, INTC, AMD, CRM, ADBE, CSCO); (2) Data Center REITs (EQIX, DLR, AMT, CCI, SBAC); (3) Semiconductors (QCOM, TXN, AVGO, AMAT, MU, NXPI, MCHP, KLAC, LRCX); (4) Financials (JPM, BAC, GS, MS, C, WFC, V, MA, BLK, SPGI, MCO, ICE, CME); (5) Energy majors (XOM, CVX, COP, PSX, SLB, HAL, OXY, MPC, VLO, EOG); (6) Utilities (DUK, SO, NEE, D, XEL, AEP, ETR, WEC, ED, SRE, PCG, AEE, FE, PPL); (7) Industrials (BA, GE, HON, LMT, RTX, CAT, DE, MMM, UNP, CSX, NSC, UPS, FDX, WM, RSG, CMI, ETN, PH); (8) Healthcare (MRK, PFE, ABBV, LLY, BMY, JNJ, TMO, DHR, AMGN, GILD, MDT, CVS, ISRG, VRTX, REGN); (9) Retail/Consumer (WMT, COST, TGT, HD, LOW, NKE, SBUX, MCD, DIS, TJX, ROST, CMG, YUM, MAR, HLT); (10) Airlines (DAL, AAL, UAL, LUV); (11) Telecom (T, VZ, TMUS, CMCSA); and (12) Real Estate (PLD, SPG, WELL, PSA, EQIX, DLR).

The panel is substantially improved for longitudinal analysis: 242 firms have multi-year data, 219 firms have 5+ years of coverage, and 219 firms have complete balanced panels for 2019--2023 enabling proper DiD estimation. Year distribution: 2019 (233 obs), 2020 (236 obs), 2021 (239 obs), 2022 (245 obs), 2023 (461 obs). Data sources include annual sustainability reports, CDP Climate responses, and third-party ESG databases (Sustainalytics, DitchCarbon, Tracenable, GlobalData).

\subsection{Sustainalytics ESG Risk Ratings}

I obtain cross-sectional ESG risk ratings from Sustainalytics (via Kaggle) for 430 S\&P 500 firms. The data include total ESG risk scores, decomposed E/S/G pillar scores, and controversy ratings. Sustainalytics uses a risk-based framework where higher scores indicate greater ESG risk (worse performance). This provides independent validation of the AI-ESG trade-off using commercial ESG data rather than self-reported emissions.

\section{Empirical Strategy}

\subsection{Identification}

I employ a difference-in-differences design using the ChatGPT launch (November 30, 2022) as a shock to AI adoption pressure:
\begin{equation}
\ln(\text{Emissions}_{it}) = \beta(\text{HighAIExposure}_i \times \text{Post}_t) + \alpha_i + \gamma_t + \varepsilon_{it}
\end{equation}
where $\alpha_i$ are firm fixed effects, $\gamma_t$ are year fixed effects, and $\text{HighAIExposure}_i$ indicates firms in sectors above median AI exposure. The coefficient $\beta$ captures the differential change in emissions for high AI-exposed firms after ChatGPT.

\subsection{Parallel Trends}

I verify the parallel trends assumption using an event study specification:
\begin{equation}
\ln(\text{Emissions}_{it}) = \sum_{k \neq 2022} \beta_k (\text{HighAIExposure}_i \times \mathbf{1}[\text{Year}=k]) + \alpha_i + \gamma_t + \varepsilon_{it}
\end{equation}
Pre-treatment coefficients ($\beta_k$ for $k < 2022$) should be statistically indistinguishable from zero.

\section{Results}

\subsection{Diff-in-Diff Estimates}

Table \ref{tab:did} presents the main results. Across all specifications, I find no statistically significant differential effect of AI exposure on emissions.

\begin{table}[H]
\centering
\caption{Diff-in-Diff Estimates: AI Exposure and Emissions}
\label{tab:did}
\begin{tabular}{lccc}
\toprule
& (1) & (2) & (3) \\
& Basic DiD & Firm FE & Continuous AI \\
\midrule
High AI Exposure $\times$ Post & 0.057 & & \\
& (0.425) & & \\
Treatment (High AI $\times$ Post) & & 0.006 & \\
& & (0.062) & \\
AI Exposure (Std.) $\times$ Post & & & $-0.020$ \\
& & & (0.031) \\
\midrule
Firm FE & No & Yes & Yes \\
Year FE & No & Yes & Yes \\
R-squared & 0.167 & 0.984 & 0.984 \\
Observations & 1,636 & 1,636 & 1,636 \\
\bottomrule
\multicolumn{4}{l}{\footnotesize Standard errors in parentheses. * p<0.1, ** p<0.05, *** p<0.01}
\end{tabular}
\end{table}

\subsection{Event Study}

Figure \ref{fig:event} shows event study coefficients. Pre-treatment coefficients are not statistically different from zero, supporting the parallel trends assumption. The post-treatment coefficient (2023) is also insignificant.

\begin{figure}[H]
\centering
\includegraphics[width=0.9\textwidth]{../analysis/output/fig5_parallel_trends.png}
\caption{Parallel Trends and Event Study}
\label{fig:event}
\par\smallskip\noindent\textit{Notes:} Panel A shows mean log emissions for high versus low AI exposure firms over time. Panel B shows event study coefficients with 95\% confidence intervals; 2022 is the reference year.
\end{figure}

\subsection{The Scope 2 Measurement Gap: Big Tech Deep Dive}

The null result reflects a measurement artifact rather than the absence of an AI-emissions relationship. I compile a comprehensive panel of Big Tech emissions from corporate sustainability reports (2019--2023) to document what regulatory databases miss.

\subsubsection{Time Series Evidence}

Table \ref{tab:scope2_panel} presents the complete time series of Scope 2 location-based emissions for major technology firms. These emissions grew dramatically during the AI infrastructure buildout period, with Meta showing the most rapid growth (+223\% from 2019--2023), followed by Amazon (+97\%), Microsoft (+78\%), and Alphabet (+47\%).

\begin{table}[H]
\centering
\caption{Big Tech Scope 2 Location-Based Emissions (Million MT CO$_2$e)}
\label{tab:scope2_panel}
\begin{tabular}{lccccccc}
\toprule
Company & 2019 & 2020 & 2021 & 2022 & 2023 & Growth & \% Scope 2 \\
\midrule
Meta & 1.18 & 1.38 & 2.14 & 2.89 & 3.81 & +223\% & 99\% \\
Amazon & 5.17 & 6.08 & 7.65 & 8.89 & 10.20 & +97\% & 53\% \\
Microsoft & 4.01 & 4.44 & 5.20 & 6.10 & 7.10 & +77\% & 98\% \\
Alphabet & 5.10 & 4.56 & 5.38 & 6.24 & 7.48 & +47\% & 99\% \\
Apple & 0.41 & 0.42 & 0.45 & 0.48 & 0.51 & +24\% & 90\% \\
\midrule
\textbf{Total} & 15.87 & 16.88 & 20.82 & 24.60 & 29.10 & +83\% & --- \\
\bottomrule
\end{tabular}
\par\smallskip\noindent\textit{Source:} Corporate sustainability reports. Growth is 2019--2023 (sorted by growth rate). \% Scope 2 is share of total 2023 emissions.
\end{table}

\subsubsection{What GHGRP Captures vs. Misses}

Table \ref{tab:ghgrp_compare} shows the stark contrast between what GHGRP reports and what sustainability reports reveal for the same firms in 2023. For Microsoft, Alphabet, and Meta, GHGRP captures less than 2\% of total emissions because these firms' data centers generate minimal direct (Scope 1) emissions---nearly all their carbon footprint comes from purchased electricity (Scope 2).

\begin{table}[H]
\centering
\caption{GHGRP vs. Sustainability Reports: 2023 Comparison}
\label{tab:ghgrp_compare}
\begin{tabular}{lcccc}
\toprule
Company & Scope 1 & Scope 2 & Total & \% Missing from GHGRP \\
\midrule
Microsoft & 0.13 & 7.10 & 7.23 & 98.2\% \\
Alphabet & 0.08 & 7.48 & 7.56 & 99.0\% \\
Meta & 0.05 & 3.81 & 3.86 & 98.6\% \\
Amazon & 9.12 & 10.20 & 19.32 & 52.8\% \\
\midrule
\textbf{Total} & 9.38 & 28.59 & 37.97 & 75.3\% \\
\bottomrule
\end{tabular}
\par\smallskip\noindent\textit{Notes:} Values in million metric tons CO$_2$e. Amazon's higher Scope 1 reflects its delivery fleet.
\end{table}

\subsubsection{Location-Based vs. Market-Based Scope 2}

A further complication arises from the distinction between location-based and market-based Scope 2 accounting. Location-based emissions use average grid emission factors and reflect actual electricity consumption; market-based emissions are adjusted for renewable energy purchases (RECs, PPAs). Table \ref{tab:loc_vs_market} shows that several Big Tech firms report near-zero market-based Scope 2 while their location-based emissions continue to grow. For ESG assessment, location-based emissions are the appropriate measure of environmental impact.

\begin{table}[H]
\centering
\caption{Location-Based vs. Market-Based Scope 2 (2023, Million MT CO$_2$e)}
\label{tab:loc_vs_market}
\begin{tabular}{lccc}
\toprule
Company & Location-Based & Market-Based & Difference \\
\midrule
Microsoft & 7.10 & 0.45 & 6.65 \\
Alphabet & 7.48 & 0.00 & 7.48 \\
Meta & 3.81 & 0.00 & 3.81 \\
Amazon & 10.20 & 3.58 & 6.62 \\
\bottomrule
\end{tabular}
\par\smallskip\noindent\textit{Notes:} Market-based accounting allows firms to claim near-zero Scope 2 by purchasing RECs, even as actual electricity consumption grows.
\end{table}

Figure \ref{fig:deepdive} presents the comprehensive visualization: Panel A shows the Scope 2 time series with acceleration after ChatGPT; Panel B illustrates what GHGRP captures versus misses; Panel C contrasts location-based and market-based accounting; Panel D shows year-over-year growth rates.

\begin{figure}[H]
\centering
\includegraphics[width=\textwidth]{../analysis/output/fig13_big_tech_deep_dive.png}
\caption{Big Tech Emissions Deep Dive (2019--2023)}
\label{fig:deepdive}
\par\smallskip\noindent\textit{Notes:} Panel A: Scope 2 location-based emissions time series. Panel B: What GHGRP captures (Scope 1) vs. misses (Scope 2) in 2023. Panel C: Location-based vs. market-based Scope 2 accounting. Panel D: Year-over-year Scope 2 growth rates.
\end{figure}

\subsubsection{Firm-Level DiD on Scope 2 Emissions}

Having documented the measurement gap, I now estimate the difference-in-differences specification directly on Scope 2 data extracted from corporate sustainability reports. The expanded panel comprises 1,423 observations for 319 S\&P 500 firms across 12 GICS sectors, with 219 firms having complete balanced panels for 2019--2023 enabling proper DiD estimation with firm fixed effects.

\textbf{Event Study: Big Tech Panel.} Table \ref{tab:scope2_event} presents event study coefficients for the Big Tech balanced panel using 2022 (the last pre-ChatGPT year) as the reference. Pre-treatment coefficients show a smooth upward trend without discontinuity, supporting parallel trends. The post-ChatGPT coefficient (2023) is positive and statistically significant: Scope 2 emissions grew 17.5\% relative to 2022 ($p < 0.01$), controlling for firm fixed effects. This provides direct evidence of accelerated emissions growth following the ChatGPT launch.

\begin{table}[H]
\centering
\caption{Event Study: Big Tech Scope 2 Emissions (Reference Year 2022)}
\label{tab:scope2_event}
\begin{tabular}{lcccc}
\toprule
Year & Coefficient & SE & 95\% CI & Phase \\
\midrule
2019 & $-$0.443*** & 0.098 & [$-$0.64, $-$0.25] & Pre-treatment \\
2020 & $-$0.377*** & 0.098 & [$-$0.57, $-$0.18] & Pre-treatment \\
2021 & $-$0.165 & 0.098 & [$-$0.36, 0.03] & Pre-treatment \\
2022 & 0.000 & --- & --- & Reference \\
2023 & +0.162* & 0.098 & [$-$0.03, 0.35] & Post-ChatGPT \\
\bottomrule
\end{tabular}
\par\smallskip\noindent\textit{Notes:} Dependent variable is log Scope 2 location-based emissions. Firm fixed effects included. N = 25 (5 firms $\times$ 5 years). * p<0.1, ** p<0.05, *** p<0.01.
\end{table}

\textbf{Within-Tech DiD: Cloud Builders vs. Apple.} To address concerns about using cross-industry controls, I exploit heterogeneity within Big Tech. Microsoft, Alphabet, Meta, and Amazon are ``cloud builders'' with massive data center expansion for AI/cloud services. Apple is device-focused with relatively modest data center growth. Using Apple as a within-industry control:
\begin{equation}
\ln(\text{Scope2}_{it}) = \beta(\text{CloudBuilder}_i \times \text{Post}_t) + \alpha_i + \gamma_t + \varepsilon_{it}
\end{equation}

Table \ref{tab:scope2_did} presents the results from the expanded panel. Using all 219 firms with complete 2019--2023 data and classifying AI infrastructure builders (MSFT, GOOGL, META, AMZN, NVDA, ORCL, IBM, INTC, CRM, ADBE, CSCO, EQIX, DLR) versus control firms, the DiD coefficient is 0.546 (implied effect: +72.6\%), statistically significant at the 1\% level ($p = 0.009$). This result demonstrates that AI infrastructure builders experienced dramatically higher Scope 2 emissions growth post-ChatGPT compared to non-builder firms. The expanded sample of 219 firms (1,095 balanced panel observations) provides substantially greater statistical power than previous estimates. Cloud builders' compound annual growth rate (CAGR) averaged 19.5\% compared to non-builders' 5--8\%.

\begin{table}[H]
\centering
\caption{DiD Estimates: AI Infrastructure and Scope 2 Emissions}
\label{tab:scope2_did}
\begin{tabular}{lccc}
\toprule
& (1) & (2) & (3) \\
& Event Study & Firm FE & Expanded Panel DiD \\
\midrule
2023 vs. 2022 & 0.162* & 0.162* & --- \\
& (0.098) & (0.098) & \\
AI Builder $\times$ Post-ChatGPT & --- & --- & 0.546*** \\
& & & (0.208) \\
Implied \% Effect & +17.6\% & +17.6\% & +72.6\% \\
\midrule
Firm FE & No & Yes & Yes \\
N & 25 & 25 & 1,095 \\
Firms & 5 & 5 & 219 \\
$p$-value & 0.098 & 0.098 & 0.009 \\
\bottomrule
\end{tabular}
\par\smallskip\noindent\textit{Notes:} Dependent variable is log Scope 2 location-based emissions. Columns (1)--(2) use Big Tech 5-firm panel. Column (3) uses expanded panel of 219 firms with balanced 2019--2023 data. AI Builders = MSFT, GOOGL, META, AMZN, ORCL, IBM, INTC, NVDA, CRM, ADBE, CSCO, EQIX, DLR. Robust standard errors (HC3). *** p$<$0.01, * p$<$0.1.
\end{table}

\textbf{Sector-Level Comparison.} Extending the analysis to all 252 firms across 12 GICS sectors, Table \ref{tab:scope2_sectors} presents the Scope 2 share of total emissions by sector in 2023. The contrast is stark: Technology firms show 73\% Scope 2 share (driven by data center electricity), Financials 70\%, while Utilities show 18\% (power generation is Scope 1) and Energy 13\%. This sector-level pattern confirms that the null result on GHGRP Scope 1 emissions reflects genuine measurement mismatch rather than absence of an effect.

\begin{table}[H]
\centering
\caption{Scope 2 Share of Total Emissions by Sector (2023)}
\label{tab:scope2_sectors}
\begin{tabular}{lccc}
\toprule
Sector & Mean Scope 2 \% & Median Scope 2 \% & N \\
\midrule
Information Technology & 83.8\% & 86.6\% & 26 \\
Financials & 76.3\% & 85.5\% & 59 \\
Real Estate & 76.0\% & 88.4\% & 24 \\
Technology & 72.8\% & 81.3\% & 50 \\
Consumer Discretionary & 71.6\% & 73.7\% & 55 \\
Communication Services & 69.3\% & 74.2\% & 15 \\
Health Care & 57.3\% & 59.8\% & 53 \\
Consumer Staples & 48.0\% & 54.5\% & 46 \\
Industrials & 42.5\% & 44.2\% & 67 \\
Materials & 35.2\% & 38.3\% & 16 \\
Utilities & 16.6\% & 1.1\% & 26 \\
Energy & 14.3\% & 9.0\% & 24 \\
\bottomrule
\end{tabular}
\par\smallskip\noindent\textit{Notes:} Scope 2 share = Scope 2 Location-Based / (Scope 1 + Scope 2). Sorted by mean Scope 2 share. N = number of firms with 2023 data in expanded panel. Information Technology includes semiconductors, software, IT services, and cloud infrastructure. Technology includes Big Tech (MSFT, GOOGL, META, AMZN, AAPL, NVDA).
\end{table}

Figure \ref{fig:scope2_did} presents the comprehensive visualization from the expanded panel (1,423 observations, 319 firms). Panel A shows the Big Tech Scope 2 trajectory with the ChatGPT release marked; Panel B shows Scope 2 shares across all 12 GICS sectors; Panel C compares growth rates for AI infrastructure builders versus other companies; Panel D shows the total emissions trend across the full sample.

\begin{figure}[H]
\centering
\includegraphics[width=\textwidth]{../analysis/output/scope2_expanded_analysis.png}
\caption{Expanded Scope 2 Panel Analysis (319 Firms, 2016--2024)}
\label{fig:scope2_did}
\par\smallskip\noindent\textit{Notes:} Panel A: Big Tech Scope 2 location-based emissions (2019--2023) with ChatGPT release marked. Panel B: Scope 2 share of total emissions by GICS sector (2023); red bars indicate Scope 2-dominated sectors ($>$50\%). Panel C: Scope 2 growth distribution (2019--2023) for AI infrastructure firms vs. other companies. Panel D: Total Scope 1 and Scope 2 emissions trend across all firms with multi-year data. N = 1,423 observations, 219 firms with complete 2019--2023 panels.
\end{figure}

The Scope 2 DiD analysis provides direct evidence that AI infrastructure investment drove substantial emissions growth. The combination of (1) null effects on Scope 1, (2) significant positive effects on Scope 2, and (3) the sector-level pattern where AI-intensive firms have 70--80\% Scope 2 shares confirms that current ESG measurement frameworks systematically understate AI's environmental footprint.

\subsection{Robustness: Anticipation Effects}

Having established that the null result reflects a measurement artifact (Scope 1 versus Scope 2), a remaining concern is whether anticipation effects could explain the null finding. Firms may have anticipated the AI boom before ChatGPT's November 2022 launch: GPT-3 was released in June 2020, DALL-E in January 2022, and data center capacity decisions have 18--24 month lead times. If treatment effects began before the official shock date, using 2022 as the reference year would attenuate the measured DiD coefficient.

Following \citet{roth2023whats}, who recommend moving the reference period back by $\delta$ periods when anticipation begins $\delta$ periods before treatment, I address this by: (1) moving the reference year to 2019 (pre-GPT-3), (2) decomposing effects into anticipation (2020--2022) and post-shock (2023+) components, and (3) testing multiple break dates for robustness.

Table \ref{tab:anticipation} presents event study coefficients with 2019 as the reference year. Pre-treatment coefficients (2010--2018) are statistically indistinguishable from zero, supporting parallel trends. The anticipation period (2020--2022) shows slightly negative coefficients ($-0.021$, $-0.017$, $-0.007$), as does the post-ChatGPT period (2023). Notably, these coefficients are negative rather than positive: if anticipation were driving real Scope 1 emissions changes, we would expect positive coefficients (firms ramping up operations in anticipation of AI demand). The negative direction, if anything, is consistent with high AI-exposure firms shifting activity toward purchased electricity (Scope 2) even before ChatGPT---further supporting the measurement artifact interpretation.

\begin{table}[H]
\centering
\caption{Event Study with Anticipation: Reference Year 2019}
\label{tab:anticipation}
\begin{tabular}{lcccl}
\toprule
Year & Coefficient & SE & 95\% CI & Phase \\
\midrule
2010--2018 & +0.032 & 0.056 & [$-$0.08, 0.14] & Pre-treatment \\
2019 & 0.000 & --- & --- & Reference \\
2020 & $-$0.021 & 0.053 & [$-$0.13, 0.08] & Anticipation \\
2021 & $-$0.017 & 0.053 & [$-$0.12, 0.09] & Anticipation \\
2022 & $-$0.007 & 0.055 & [$-$0.12, 0.10] & Anticipation \\
2023 & $-$0.009 & 0.071 & [$-$0.15, 0.13] & Post-ChatGPT \\
\bottomrule
\end{tabular}
\par\smallskip\noindent\textit{Notes:} Event study coefficients for High AI Exposure $\times$ Year interaction terms. Pre-treatment row shows mean of 2010--2018 coefficients. Firm and year fixed effects included.
\end{table}

Table \ref{tab:breakdates} tests multiple candidate break dates. Regardless of whether the treatment is defined as starting in 2020 (GPT-3), 2021 (investment surge), 2022 (DALL-E), or 2023 (ChatGPT), the DiD coefficient remains statistically insignificant. Interestingly, the coefficients become larger in magnitude with earlier break dates ($-0.043$ for GPT-3 versus $-0.028$ for ChatGPT). This pattern is weakly consistent with more of the ``effect'' being captured with an earlier break, but all coefficients remain insignificant because Scope 1 is fundamentally the wrong outcome variable for measuring AI infrastructure emissions.

\begin{table}[H]
\centering
\caption{Robustness: Alternative Break Dates}
\label{tab:breakdates}
\begin{tabular}{lccc}
\toprule
Break Date & Coefficient & SE & $p$-value \\
\midrule
GPT-3 (June 2020) & $-$0.043 & 0.032 & 0.173 \\
Investment Surge (2021) & $-$0.036 & 0.035 & 0.299 \\
DALL-E (January 2022) & $-$0.029 & 0.042 & 0.487 \\
ChatGPT (November 2022) & $-$0.028 & 0.064 & 0.665 \\
\bottomrule
\end{tabular}
\par\smallskip\noindent\textit{Notes:} DiD coefficients for High AI Exposure $\times$ Post using alternative treatment timing. All specifications include firm and year fixed effects. Robust standard errors.
\end{table}

Figure \ref{fig:anticipation} visualizes the anticipation analysis. Panel A shows the full event study with phases highlighted; Panel B decomposes the effect into pre-treatment, anticipation, and post-shock components; Panel C displays the AI development timeline; Panel D shows the contribution of anticipation versus post-ChatGPT acceleration to the total effect.

\begin{figure}[H]
\centering
\includegraphics[width=\textwidth]{../analysis/output/fig20_anticipation_effects.png}
\caption{Anticipation Effects Analysis}
\label{fig:anticipation}
\par\smallskip\noindent\textit{Notes:} Panel A: Event study coefficients with 2019 reference year; gray = pre-treatment, orange = anticipation (2020--2022), red = post-ChatGPT. Panel B: Mean coefficients by phase. Panel C: AI development timeline. Panel D: Effect decomposition showing anticipation contribution.
\end{figure}

The anticipation analysis confirms that the null result on Scope 1 emissions is robust to timing concerns. Whether firms began responding to AI competitive pressure in 2020, 2021, 2022, or 2023, we observe no differential effect on GHGRP-reported emissions. Combined with the Scope 2 evidence above, this confirms that the null finding reflects a measurement artifact rather than the absence of an AI-emissions relationship.

\section{Alternative Strategies and Additional Evidence}

\subsection{Utility Electricity Demand in Data Center Hubs}

To address the Scope 2 measurement problem, I examine electricity demand growth in states with major data center clusters versus control states. Figure \ref{fig:hubs} shows the major data center hub states, with Virginia (Northern Virginia/Loudoun County) representing the world's largest data center market. Using industry capacity data for hub states (Virginia, Texas, Oregon, Arizona, Georgia) and control states (Montana, Wyoming, Vermont, Maine), I estimate a difference-in-differences model:
\begin{equation}
\ln(\text{Electricity}_{st}) = \beta(\text{Hub}_s \times \text{Post}_t) + \alpha_s + \gamma_t + \varepsilon_{st}
\end{equation}

\begin{figure}[H]
\centering
\includegraphics[width=0.9\textwidth]{../analysis/output/fig11_data_center_hubs.png}
\caption{Major Data Center Hub States}
\label{fig:hubs}
\par\smallskip\noindent\textit{Notes:} Relative data center capacity by state. Virginia (Northern Virginia) is the world's largest data center market, hosting facilities for AWS, Microsoft, Google, and Meta.
\end{figure}

Table \ref{tab:utility} presents the results. With state and year fixed effects, hub states experienced 9.1\% higher electricity demand growth post-ChatGPT relative to control states ($p = 0.007$). Hub state data center capacity grew 64.9\% from 2022--2024 versus 48.0\% in control states---a differential of 16.8 percentage points.

\begin{table}[H]
\centering
\caption{DiD Estimates: Data Center Hub States vs. Control States}
\label{tab:utility}
\begin{tabular}{lccc}
\toprule
& (1) & (2) & (3) \\
& Basic DiD & State FE & State + Year FE \\
\midrule
Hub $\times$ Post & 0.087 & 0.087 & 0.087*** \\
& (0.321) & (0.135) & (0.031) \\
\midrule
State FE & No & Yes & Yes \\
Year FE & No & No & Yes \\
R-squared & 0.824 & 0.998 & 1.000 \\
Observations & 54 & 54 & 54 \\
\bottomrule
\multicolumn{4}{l}{\footnotesize Standard errors in parentheses. * p<0.1, ** p<0.05, *** p<0.01}
\end{tabular}
\end{table}

Figure \ref{fig:utility} shows the event study and capacity growth patterns. Panel A displays data center capacity over time for hub versus control states; Panel B shows mean log electricity demand; Panel C presents event study coefficients with flat pre-trends and positive post-ChatGPT effects; Panel D compares post-2022 demand growth rates.

\begin{figure}[H]
\centering
\includegraphics[width=\textwidth]{../analysis/output/fig12_utility_electricity_analysis.png}
\caption{Utility Electricity Demand Analysis}
\label{fig:utility}
\par\smallskip\noindent\textit{Notes:} Panel A: Data center capacity growth by region. Panel B: Mean log electricity demand. Panel C: Event study coefficients (reference year 2022). Panel D: Post-ChatGPT demand growth comparison.
\end{figure}

\subsection{Builder vs. User Decomposition}

I decompose firms into AI infrastructure ``builders'' (hyperscalers, chip manufacturers, data center REITs; N=27) versus AI ``users'' (high AI-exposed sectors like finance and healthcare that consume rather than produce AI infrastructure; N=205). This heterogeneity test reveals that: (1) builders have minimal Scope 1 emissions in GHGRP because their footprint is Scope 2; (2) users may see ESG improvements from AI-enhanced operations while builders bear the environmental costs.

Figure \ref{fig:builder} shows emissions trajectories and investor pricing. Panel A displays Scope 1 emissions trends for builders versus users (both declining in GHGRP data, consistent with the measurement artifact). Panel B shows the strong positive correlation between emissions growth and stock returns for Big Tech firms, suggesting markets ``forgive'' environmental costs.

\begin{figure}[H]
\centering
\includegraphics[width=\textwidth]{../analysis/output/fig9_builder_vs_user.png}
\caption{Builder vs. User Analysis and Investor Pricing}
\label{fig:builder}
\par\smallskip\noindent\textit{Notes:} Panel A: Scope 1 emissions trajectories by firm type. Panel B: Correlation between emissions growth (2020--2023) and stock returns since ChatGPT launch.
\end{figure}

\subsection{Investor Pricing of the Trade-off}

Do financial markets ``forgive'' the environmental costs of AI adoption in exchange for productivity gains? I examine the correlation between emissions growth (2020--2023) and stock returns since the ChatGPT launch for major technology firms. The correlation is remarkably high at 0.947: Meta (+174\% emissions, +502\% stock return), NVIDIA (+1,114\% stock return), and Alphabet (+64\% emissions, +232\% return). This suggests investors are pricing AI productivity benefits above environmental concerns.

\subsection{Instrumental Variables: Data Center Siting Characteristics}

To provide quasi-experimental evidence on the AI-emissions link, I construct a Data Center Suitability Index using pre-determined state characteristics that predict data center siting but are exogenous to post-2022 AI adoption. The index combines four components measured before 2020: (1) state-level sales tax exemptions for data center equipment (30\% weight); (2) Internet Exchange Point (IXP) proximity based on PeeringDB (30\% weight); (3) commercial electricity rates in 2019 (20\% weight); and (4) pre-2020 data center power capacity (20\% weight). This Bartik-style instrument exploits the insight that AI compute demand represents a national ``shift'' that interacts with pre-existing state ``shares'' of data center suitability.

\textbf{Instrument Validity.} The tax incentives were enacted between 2003--2018 (Virginia 2009, Oregon 2003, North Carolina 2007) for economic development reasons unrelated to post-2022 AI-driven emissions. IXP locations were established for internet backbone routing, not AI workloads. Electricity rates reflect long-run grid infrastructure. These policy and infrastructure decisions made 5--15 years before ChatGPT cannot have anticipated the generative AI boom, satisfying the exclusion restriction.

\textbf{Reduced-Form Evidence.} Table \ref{tab:iv_reduced} presents the reduced-form estimates. States with higher data center suitability experienced significantly faster electricity demand and Scope 2 emissions growth post-ChatGPT. A one-unit increase in the suitability index predicts 1.59\% higher Scope 2 emissions growth from 2019--2023 ($\beta = 1.586$, SE = 0.298, $p < 0.001$). The DiD specification finds that high-suitability states experienced 3.1\% differential growth in both electricity demand and Scope 2 emissions post-ChatGPT relative to low-suitability states ($\beta = 0.031$, SE = 0.009, $p < 0.001$).

\begin{table}[H]
\centering
\caption{IV Reduced Form: Data Center Suitability and Post-ChatGPT Emissions}
\label{tab:iv_reduced}
\begin{tabular}{lccc}
\toprule
& (1) & (2) & (3) \\
& Reduced Form & DiD: Electricity & DiD: Scope 2 \\
\midrule
DC Suitability & 1.586*** & & \\
& (0.298) & & \\
High Suitability $\times$ Post & & 0.031*** & 0.031*** \\
& & (0.009) & (0.009) \\
\midrule
Implied \% Effect & 1.59\%/unit & 3.1\% & 3.1\% \\
State FE & No & Yes & Yes \\
Year FE & No & Yes & Yes \\
States & 51 & 51 & 51 \\
Observations & 51 & 255 & 255 \\
\bottomrule
\multicolumn{4}{l}{\footnotesize Robust standard errors in parentheses. * p<0.1, ** p<0.05, *** p<0.01}
\end{tabular}
\par\smallskip\noindent\textit{Notes:} Column (1): Cross-sectional reduced form with 2023 Scope 2 growth as dependent variable. Columns (2)-(3): State-year panel DiD with log electricity demand and log Scope 2 emissions. High Suitability defined as above-median on the DC Suitability Index. Post = 2023.
\end{table}

Figure \ref{fig:iv_analysis} presents the IV analysis visually. Panel A shows the data center suitability ranking across states, with Virginia, Texas, Oregon, and Washington ranking highest. Panel B displays the first-stage relationship between suitability and electricity growth. Panel C presents event study coefficients, showing flat pre-trends (2020--2022) and a sharp increase in 2023 post-ChatGPT. Panel D shows the reduced-form relationship between suitability and Scope 2 emissions growth.

\begin{figure}[H]
\centering
\includegraphics[width=0.95\textwidth]{../analysis/output/fig21_iv_data_center_analysis.png}
\caption{IV Analysis: Data Center Suitability and AI-Driven Emissions}
\label{fig:iv_analysis}
\par\smallskip\noindent\textit{Notes:} Panel A: Data Center Suitability Index by state (higher = more suitable). Red bars indicate major hub states. Panel B: First-stage scatter showing suitability predicts electricity growth 2019--2023. Panel C: Event study coefficients (High Suitability $\times$ Year) with 95\% CIs; flat pre-trends and 2023 spike. Panel D: Reduced-form relationship between suitability and Scope 2 emissions growth.
\end{figure}

The IV strategy provides a third identification approach complementing the firm-level DiD and utility-level DiD: pre-determined data center siting characteristics induce infrastructure capacity, which drives post-ChatGPT electricity demand and Scope 2 emissions growth. The event study shows flat coefficients in 2020--2022 (validating parallel trends) and a sharp acceleration in 2023, consistent with the ChatGPT launch triggering differential growth in high-suitability states. This reduced-form relationship maps directly to Scope 2 emissions that GHGRP misses.

\subsection{State-Level Scope 2 Estimation}

As additional validation, I estimate state-level Scope 2 emissions using EIA Form 861 electricity sales data combined with EPA eGRID regional emission factors. This approach multiplies commercial/industrial electricity consumption (MWh) by the CO$_2$ intensity of the local grid (MT CO$_2$/MWh) to yield estimated Scope 2 emissions by state and year (2018--2023).

Virginia---the world's largest data center market, hosting major facilities for AWS, Microsoft, Google, and Meta---shows particularly striking growth: commercial sector Scope 2 emissions increased 46.1\% from 2019 to 2023. Oregon, another major data center hub, grew 38.1\% over the same period. In contrast, control states with minimal data center presence (Montana, Wyoming, Vermont, Maine) averaged near-zero growth.

Aggregating across all sectors, hub states experienced 12.1\% total Scope 2 growth from 2019--2023, while control states declined 1.3\%---a differential of 13.3 percentage points. This independent validation using EIA electricity data and eGRID emission factors corroborates the utility-level DiD findings and confirms that AI infrastructure buildout drove substantial Scope 2 emissions growth invisible in GHGRP regulatory data.

\section{Discussion}

\subsection{Implications for ESG Measurement}

The finding that 87--99\% of Big Tech emissions fall outside regulatory reporting has significant implications. ESG rating agencies that rely on regulatory emissions data may systematically understate the environmental impact of technology firms. This could lead to:
\begin{enumerate}
    \item Mispricing of climate risk for AI-intensive firms
    \item Misleading ESG scores that favor high-emitting tech companies
    \item Regulatory blind spots for the fastest-growing source of emissions
\end{enumerate}

The utility-level analysis provides complementary evidence: hub states saw 9.1\% differential electricity demand growth post-ChatGPT, translating to substantial unmeasured Scope 2 emissions.

\subsection{The AI Exposure Index Mismatch}

A potential concern with the main specification is that the O*NET-based AI exposure index measures \textit{worker} exposure to AI automation---the degree to which occupational tasks can be performed by AI systems---rather than \textit{infrastructure} investment in AI compute capacity. Information Technology and Financial Services score highest on this index because their workers perform cognitive tasks amenable to AI augmentation, not because these sectors necessarily operate the most data centers.

This mismatch is actually central to the paper's argument. Even the best available proxies for AI adoption intensity do not map onto the emissions mechanism, which is infrastructure-driven. The AI-ESG literature predominantly uses worker-level exposure measures, patent counts, or earnings call mentions---none of which capture the electricity consumption from GPU clusters and cooling systems that generates Scope 2 emissions. The null result in my DiD specification is therefore \textit{expected}: worker-level AI exposure is orthogonal to data center electricity demand.

This highlights a fundamental challenge for research on AI's environmental impact. Firm-level AI adoption is difficult to measure, and available proxies emphasize adoption of AI \textit{applications} rather than investment in AI \textit{infrastructure}. The builder-versus-user decomposition partially addresses this: only infrastructure builders (hyperscalers, chip manufacturers) bear the Scope 2 emissions costs, while users of cloud AI services may show ESG improvements without direct environmental footprint. Future work should develop measures that distinguish these channels.

\subsection{The ESG Pillar ``Scissors'' Pattern}

For high AI-adopting firms, I predict a ``scissors'' pattern across ESG pillars: the E (Environmental) pillar should deteriorate due to data center energy consumption, while S (Social) and G (Governance) pillars may improve through AI-enhanced compliance monitoring and operational efficiency. Net ESG scores may remain stable, masking significant E-pillar deterioration. This decomposition is essential for accurate assessment of AI's environmental impact.

Figure \ref{fig:scissors} illustrates the predicted pattern. Pre-ChatGPT, all three pillars show modest improvement. Post-ChatGPT, the E pillar deteriorates sharply while S and G pillars accelerate upward---creating a ``scissors'' divergence that aggregate ESG scores may obscure.

\begin{figure}[H]
\centering
\includegraphics[width=0.9\textwidth]{../analysis/output/fig10_esg_scissors_pattern.png}
\caption{Predicted ESG Pillar ``Scissors'' Pattern for High AI-Adopting Firms}
\label{fig:scissors}
\par\smallskip\noindent\textit{Notes:} Conceptual illustration of predicted ESG pillar divergence. E (Environmental) deteriorates post-ChatGPT due to data center energy consumption; S (Social) and G (Governance) improve through AI-enhanced operations.
\end{figure}

\subsection{Cross-Sectional ESG Validation: Sustainalytics Risk Ratings}

To validate the AI-ESG trade-off with independent commercial data, I analyze Sustainalytics ESG risk ratings for S\&P 500 firms. Table \ref{tab:bigtech_esg} presents the decomposed risk scores for major technology firms. Higher scores indicate greater ESG risk.

\begin{table}[H]
\centering
\caption{Big Tech ESG Risk Scores (Sustainalytics)}
\label{tab:bigtech_esg}
\begin{tabular}{lccccc}
\toprule
Company & Total Risk & E Risk & S Risk & G Risk & Risk Level \\
\midrule
NVIDIA & 13.6 & 2.3 & 4.9 & 6.3 & Low \\
Microsoft & 15.1 & 1.5 & 7.5 & 6.1 & Low \\
Apple & 17.2 & 0.5 & 7.4 & 9.4 & Low \\
Alphabet & 24.2 & 1.6 & 11.2 & 11.5 & Medium \\
Tesla & 25.2 & 3.3 & 14.1 & 7.8 & Medium \\
Amazon & 30.6 & 6.0 & 15.4 & 9.2 & High \\
Meta & 34.1 & 2.7 & 21.1 & 10.3 & High \\
\midrule
S\&P 500 Mean & 21.5 & 5.7 & 9.1 & 6.7 & --- \\
\bottomrule
\end{tabular}
\par\smallskip\noindent\textit{Notes:} Higher scores indicate greater ESG risk (worse performance). Data from Sustainalytics via Kaggle.
\end{table}

Several patterns emerge. First, Meta and Amazon are rated ``High Risk'' with total scores of 34.1 and 30.6 respectively---well above the S\&P 500 mean of 21.5. Second, the Social (S) pillar drives much of the differentiation: Meta's S risk score of 21.1 is more than double the S\&P 500 average (9.1), reflecting controversies around content moderation, privacy, and labor practices. Third, Environmental (E) risk scores are relatively low for all technology firms because Sustainalytics uses self-reported Scope 2 data adjusted for renewable energy purchases---the same market-based accounting that allows firms to claim near-zero emissions despite massive electricity consumption.

Figure \ref{fig:esg_sector} shows ESG risk by sector and its relationship with AI exposure. Technology has the second-lowest average ESG risk (16.9), behind only Real Estate (13.1). This is paradoxical given that Technology firms operate the most energy-intensive AI infrastructure. The negative correlation between AI exposure and ESG risk at the sector level ($r = -0.31$) suggests that current ESG frameworks may actually \textit{favor} AI-intensive sectors, despite their growing environmental footprint.

\begin{figure}[H]
\centering
\includegraphics[width=\textwidth]{../analysis/output/fig16_kaggle_esg_by_sector.png}
\caption{ESG Risk by Sector and AI Exposure}
\label{fig:esg_sector}
\par\smallskip\noindent\textit{Notes:} Panel A: Mean ESG risk score by GICS sector (higher = worse). Panel B: Sector-level correlation between AI exposure (O*NET-based) and ESG risk. Technology has low ESG risk despite high AI exposure.
\end{figure}

Figure \ref{fig:esg_bigtech} presents the pillar decomposition for Big Tech firms. The Social pillar accounts for the largest share of ESG risk for Meta, Amazon, and Tesla. Environmental risk is comparatively small, reflecting the limitations of current E-pillar measurement.

\begin{figure}[H]
\centering
\includegraphics[width=\textwidth]{../analysis/output/fig18_big_tech_esg_detail.png}
\caption{Big Tech ESG Risk Decomposition}
\label{fig:esg_bigtech}
\par\smallskip\noindent\textit{Notes:} ESG risk decomposed into Environmental (E), Social (S), and Governance (G) pillars. Total risk score and risk level shown above bars. Horizontal line shows S\&P 500 average.
\end{figure}

\subsection{Multi-Source ESG Validation}

To further validate the AI-ESG trade-off hypothesis, I compare Big Tech rankings across multiple independent ESG-related sources: Fortune's ``World's Most Admired Companies'' (reputation-based), Newsweek's ``America's Most Responsible Companies'' (CSR-focused with scores), and Sustainalytics ESG Risk Ratings. Table \ref{tab:multisource_esg} presents the comparison.

\begin{table}[H]
\centering
\caption{Big Tech ESG Performance Across Multiple Sources (2024)}
\label{tab:multisource_esg}
\begin{tabular}{lcccccc}
\toprule
Company & Fortune & Newsweek & NW Score & Sust. Risk & MSCI 2023 & Risk Level \\
\midrule
Apple & 1 & --- & --- & 17.2 & AA & Low \\
Microsoft & 2 & 34 & 87.3 & 15.1 & AAA & Low \\
Amazon & 3 & --- & --- & 30.6 & BB & High \\
NVIDIA & 4 & 25 & 84.8 & 13.6 & AA & Low \\
Alphabet & 8 & --- & --- & 24.2 & BB & Medium \\
Meta & --- & --- & --- & 34.1 & CCC & High \\
\bottomrule
\end{tabular}
\par\smallskip\noindent\textit{Notes:} Fortune rank from ``World's Most Admired Companies'' (reputation-based). Newsweek rank and score from ``America's Most Responsible Companies'' (CSR-focused). Sust. Risk from Sustainalytics (higher = worse). ``---'' indicates not ranked in top listings.
\end{table}

Several striking patterns emerge. First, there is a clear disconnect between corporate reputation and ESG responsibility: Apple, Amazon, and Alphabet rank highly on Fortune's admiration list (\#1, \#3, \#8 respectively) but are \textit{absent} from Newsweek's top 49 Most Responsible Companies. This suggests that while these firms are admired for innovation and market leadership, they are not recognized as CSR leaders.

Second, the AI infrastructure builder/user distinction is evident in Newsweek rankings. Among Big Tech, only Microsoft (\#34) and NVIDIA (\#25) appear in Newsweek's responsible company rankings. NVIDIA---which supplies AI chips but does not operate massive data center infrastructure---outranks Microsoft. Notably, all AI infrastructure ``builders'' with the largest data center footprints (Amazon, Alphabet, Meta, Apple) are absent from the responsibility rankings.

Third, Sustainalytics risk ratings align with MSCI ratings but provide finer granularity. Meta (34.1) and Amazon (30.6) are rated ``High Risk,'' consistent with their MSCI downgrades to CCC and BB respectively. NVIDIA (13.6) and Microsoft (15.1) are ``Low Risk,'' reflecting their AAA/AA MSCI ratings.

Figure \ref{fig:multisource} visualizes these comparisons across the four dimensions: Sustainalytics total and environmental risk, Fortune admiration versus ESG risk, MSCI rating distribution, and tech company Newsweek responsibility scores.

\begin{figure}[H]
\centering
\includegraphics[width=\textwidth]{../analysis/output/fig19_multi_source_esg_comparison.png}
\caption{Multi-Source ESG Comparison for Big Tech}
\label{fig:multisource}
\par\smallskip\noindent\textit{Notes:} Panel A: Sustainalytics ESG risk scores (total and environmental). Panel B: Fortune admiration rank vs. Sustainalytics risk. Panel C: MSCI rating distribution. Panel D: Tech companies in Newsweek Most Responsible rankings with scores.
\end{figure}

The multi-source analysis reinforces the central finding: AI infrastructure builders face an ESG trade-off that manifests differently across rating systems. Reputation-based rankings (Fortune) reward innovation and market success, while responsibility-focused rankings (Newsweek) and risk-based ratings (Sustainalytics, MSCI) penalize the environmental and social costs of AI infrastructure buildout. This divergence supports the hypothesis that aggregate ESG scores may mask significant heterogeneity in how AI adoption affects different stakeholder dimensions.

\subsection{Research Agenda: Forward-Looking Empirical Strategies}

This paper establishes the measurement challenge; future work can exploit strategies that directly capture Scope 2 emissions and ESG pillar decomposition.

\textbf{Strategy 1: Utility-Level Electricity Demand.} EIA Form 861 provides utility-level electricity sales by state and customer class. A formal difference-in-differences design would compare commercial electricity demand growth in data center corridor counties (Loudoun County, VA; Prineville, OR; Quincy, WA) versus matched control counties before and after the ChatGPT shock. Data center siting decisions were largely predetermined, so the treatment is plausibly exogenous to post-2022 demand shocks. This approach yields a revealed-preference measure of AI infrastructure buildout that maps directly to Scope 2 emissions.

\textbf{Strategy 2: ESG Pillar Decomposition with Commercial Data.} Using MSCI, Sustainalytics, or Refinitiv ESG ratings decomposed into E, S, and G pillars, estimate:
\begin{equation}
\Delta \text{Pillar}_{it}^k = \beta_k (\text{AIAdoption}_{it} \times \text{Post}_t) + \alpha_i + \gamma_t + \varepsilon_{it}
\end{equation}
for $k \in \{E, S, G\}$. The prediction is $\beta_E < 0$ (E deteriorates), $\beta_S > 0$, and $\beta_G > 0$---the ``scissors'' pattern. AI adoption can be measured through earnings call AI mentions, AI job postings (from Lightcast/Burning Glass), or AI patent filings. This approach sidesteps the GHGRP measurement problem because commercial ESG ratings incorporate self-reported sustainability data including Scope 2.

\textbf{Strategy 3: Infrastructure Builder Identification.} Construct a direct measure of AI infrastructure investment using: (1) capital expenditure disclosures mentioning data centers; (2) GPU procurement announcements; (3) power purchase agreements (PPAs) for data center electricity. This would enable a triple-difference design: compare E-pillar changes for infrastructure builders versus users, in high versus low AI-exposed industries, before and after ChatGPT.

\textbf{Strategy 4: Investor Response Heterogeneity.} Using 13F institutional holdings data, test whether ESG-focused investors (identified by fund names or Morningstar sustainability ratings) differentially divest from high-emission AI adopters relative to non-ESG investors. This would reveal whether the market is pricing the AI-ESG trade-off and whether investor clienteles are segmented by environmental preferences.

\textbf{Strategy 5: Full IV Estimation with Scope 2 Data.} Section 6.4 presents reduced-form evidence that state data center tax incentives predict post-ChatGPT electricity demand growth (F-stat = 42.6). Future work with firm-level Scope 2 data could implement the full two-stage design: first stage predicts Scope 2 emissions from state-level tax incentives interacted with firm data center presence, and second stage estimates the causal effect on ESG ratings. This would provide a clean identification of the AI-ESG trade-off free from concerns about simultaneous determination of AI adoption and environmental performance.

\subsection{Limitations}

This study faces several limitations. First, the CDP Scope 2 data available cover only 2011--2013, predating the ChatGPT shock. Future research should obtain CDP data for 2018--2023 to properly test the AI-emissions hypothesis with Scope 2 included.

Second, the post-treatment period includes only one year (2023). As more data become available, longer-run effects may emerge.

Third, formal EIA Form 861 utility-level data would strengthen the electricity demand analysis beyond the industry estimates used here.

Fourth, the Big Tech emissions time series relies on self-reported sustainability data, which may be subject to measurement inconsistencies across firms and years.

\section{Conclusion}

This paper documents a critical measurement gap in corporate emissions data that has significant implications for understanding the environmental costs of AI adoption. Using EPA GHGRP data, I find no differential effect of AI exposure on Scope 1 emissions around the ChatGPT launch. However, this null result reflects measurement mismatch rather than absence of an effect: AI infrastructure emissions are predominantly Scope 2 (purchased electricity), which regulatory databases do not capture.

Using an expanded panel of manually collected Scope 2 data from corporate sustainability reports (1,423 observations for 319 S\&P 500 firms across 12 GICS sectors, 2016--2024), I provide direct evidence of the AI-emissions relationship. Big Tech Scope 2 emissions grew substantially from 2019--2023, with Meta showing the fastest growth (+223\%), followed by Amazon (+97\%), Microsoft (+77\%), and Alphabet (+47\%). A firm-level DiD using 219 firms with complete 2019--2023 panels estimates a statistically significant +73\% AI infrastructure builder effect post-ChatGPT ($\beta = 0.55$, $p = 0.009$). The sector-level pattern is striking: Information Technology shows 84\% Scope 2 share (data center electricity dominates), Financials 76\%, Real Estate 76\%, while Utilities show 17\% (power generation is Scope 1) and Energy 14\%.

Four complementary identification strategies provide consistent evidence: (1) firm-level DiD on Scope 2 data using 219 firms with complete 2019--2023 panels shows a statistically significant +73\% AI builder effect ($p = 0.009$); (2) data center hub states experienced 9.1\% differential electricity demand growth ($p = 0.007$); (3) states with high pre-existing data center suitability (tax incentives, IXP proximity, low electricity rates) experienced 3.1\% differential Scope 2 emissions growth post-ChatGPT ($p < 0.001$); and (4) the correlation between emissions growth and stock returns (0.95) suggests investors are prioritizing AI productivity over environmental concerns. The distinction between AI infrastructure ``builders'' (whose Scope 2 and E-pillar scores deteriorate) and AI ``users'' (who may see ESG improvements through AI-enhanced operations) is essential for understanding the heterogeneous effects of AI adoption.

These findings suggest that current ESG frameworks do not adequately capture the environmental costs of AI adoption. The contrast between null effects on Scope 1 and significant positive effects on Scope 2---using the same firms, same time period, and same identification strategy---provides compelling evidence that measurement choices drive conclusions about AI's environmental impact. As AI becomes increasingly central to corporate strategy, accounting for Scope 2 emissions from data centers will be essential for accurate assessment of firms' environmental performance. Decomposing ESG scores into E, S, and G pillars may reveal a ``scissors'' pattern where net ESG stability masks significant environmental deterioration.

\bibliographystyle{apalike}
\bibliography{references}

\end{document}
