\documentclass[12pt,a4paper]{article}

% Packages
\usepackage[utf8]{inputenc}
\usepackage[T1]{fontenc}
\usepackage{amsmath,amssymb}
\usepackage{graphicx}
\usepackage{booktabs}
\usepackage{natbib}
\usepackage{hyperref}
\usepackage{geometry}
\usepackage{setspace}
\usepackage{float}
\usepackage{caption}
\usepackage{subcaption}

\geometry{margin=1in}
\doublespacing

% Title
\title{AI Adoption and the ESG Trade-off: Evidence from Corporate Emissions Data}
\author{
    Alina Malkova\\
    \textit{Florida Institute of Technology}\\
    \texttt{amalkova@fit.edu}
}
\date{\today}

\begin{document}

\maketitle

\begin{abstract}
Does artificial intelligence adoption create a productivity-ESG trade-off for public firms? Using the ChatGPT launch (November 2022) as an exogenous shock to AI adoption pressure, I employ a difference-in-differences design to examine whether firms in high AI-exposed industries experienced differential changes in greenhouse gas emissions. Matching EPA GHGRP facility-level emissions data to S\&P 500 firms (2010--2023) and constructing an AI exposure index from O*NET occupational data, I find no significant differential effect of AI exposure on Scope 1 emissions. However, I document a critical measurement limitation: GHGRP captures only direct emissions (Scope 1), while AI infrastructure emissions are primarily Scope 2 (purchased electricity). Big Tech firms' Scope 2 emissions---which grew 60--176\% from 2020--2023---represent 87--99\% of their carbon footprint and are entirely absent from regulatory emissions databases. To address this measurement gap, I examine utility-level electricity demand in data center hub states, finding 9.1\% differential growth post-ChatGPT ($p = 0.007$). Financial markets appear to ``forgive'' these environmental costs: emissions growth and stock returns are correlated at 0.95 for major technology firms. These findings highlight that current ESG measurement frameworks systematically understate the environmental costs of AI adoption.

\medskip
\noindent\textbf{Keywords:} Artificial Intelligence, ESG, Carbon Emissions, Scope 2, Data Centers, ChatGPT

\noindent\textbf{JEL Codes:} G30, Q54, O33, M14
\end{abstract}

\newpage

\section{Introduction}

The rapid adoption of artificial intelligence technologies has created a fundamental tension for corporate environmental sustainability. Firms that aggressively deploy AI---and by extension, the data centers with massive energy footprints that power these systems---may boost productivity and competitiveness but simultaneously increase their carbon emissions. This tension is particularly acute given that environmental, social, and governance (ESG) performance increasingly matters for institutional investors, index inclusion, and cost of capital.

This paper investigates whether AI adoption creates a measurable productivity-ESG trade-off for public firms. I exploit the launch of ChatGPT in November 2022 as an exogenous shock that dramatically increased competitive pressure to adopt AI across industries. Using a difference-in-differences framework, I compare emissions trajectories of firms in high versus low AI-exposed industries before and after this shock.

My analysis reveals a critical measurement challenge: the EPA's Greenhouse Gas Reporting Program (GHGRP)---the primary regulatory source of U.S. corporate emissions data---captures only Scope 1 emissions (direct combustion from owned facilities). However, AI infrastructure emissions are predominantly Scope 2 (indirect emissions from purchased electricity to power data centers). I document that for major technology firms, Scope 2 represents 87--99\% of total emissions, and these emissions grew 60--176\% between 2020 and 2023. None of this growth appears in GHGRP data.

This measurement gap has significant implications for both academic research and policy. Studies relying on regulatory emissions databases may systematically underestimate the environmental costs of AI adoption. Current ESG frameworks may not adequately capture emissions from the fastest-growing source of corporate carbon footprints.

\section{Related Literature}

\subsection{AI Adoption and ESG Performance}

The dominant finding in recent literature is that AI adoption \textit{improves} firm-level ESG performance. Studies using Chinese A-share listed firms find positive effects across all three ESG pillars, with mechanisms including better internal controls, financing constraint alleviation, and green innovation \citep[see][for a review]{chen2025ai}. However, this literature treats AI as a tool for ESG management rather than examining the equilibrium trade-off firms face when AI adoption itself carries environmental costs.

\subsection{Big Tech's Emissions Crisis}

Emerging evidence contradicts the optimistic view for firms building AI infrastructure. Alphabet's emissions rose nearly 50\% since 2019; Meta's location-based emissions more than doubled; Microsoft's rose 23.4\% since 2020---all driven by data center electricity demand. In 2025, Microsoft, Google, Amazon, and Meta are projected to spend a combined \$320 billion on AI infrastructure. A Harvard study found that the carbon intensity of electricity used by data centers was 48\% higher than the U.S. average.

\subsection{Emissions Scope and Measurement}

Corporate emissions are classified into three scopes: Scope 1 (direct emissions from owned sources), Scope 2 (indirect emissions from purchased energy), and Scope 3 (all other indirect emissions in the value chain). For technology firms, Scope 2 dominates because data centers consume massive amounts of electricity but generate minimal direct emissions. The EPA GHGRP requires reporting only for facilities emitting more than 25,000 metric tons CO$_2$e of direct (Scope 1) emissions, creating a systematic gap for electricity-intensive operations.

\section{Data}

\subsection{EPA GHGRP Emissions Panel}

I obtain facility-level emissions data from the EPA Greenhouse Gas Reporting Program for 2010--2023. Matching facilities to parent companies using EPA ownership data and then to S\&P 500 constituents, I construct a panel of 121 firms with 1,636 company-year observations. The panel is well-balanced: 109 firms have complete data for all 14 years.

\subsection{AI Exposure Index}

Following \citet{felten2021occupational}, I construct an AI exposure index using O*NET occupational ability and work activity data. I identify abilities where AI systems have strong capabilities (e.g., written comprehension, deductive reasoning, information processing) and weight occupations by their reliance on these abilities. Aggregating to GICS sectors, I find Information Technology (81.5) and Financials (81.2) have the highest AI exposure, while Utilities (42.0) and Materials (37.2) have the lowest.

\subsection{CDP Scope 2 Data}

To examine the measurement gap, I obtain corporate emissions data from CDP (formerly Carbon Disclosure Project) for 2011--2013, which includes both Scope 1 and Scope 2 emissions. I also compile Big Tech emissions from corporate sustainability reports for 2020--2023.

\subsection{Sustainalytics ESG Risk Ratings}

I obtain cross-sectional ESG risk ratings from Sustainalytics (via Kaggle) for 430 S\&P 500 firms. The data include total ESG risk scores, decomposed E/S/G pillar scores, and controversy ratings. Sustainalytics uses a risk-based framework where higher scores indicate greater ESG risk (worse performance). This provides independent validation of the AI-ESG trade-off using commercial ESG data rather than self-reported emissions.

\section{Empirical Strategy}

\subsection{Identification}

I employ a difference-in-differences design using the ChatGPT launch (November 30, 2022) as a shock to AI adoption pressure:
\begin{equation}
\ln(\text{Emissions}_{it}) = \beta(\text{HighAIExposure}_i \times \text{Post}_t) + \alpha_i + \gamma_t + \varepsilon_{it}
\end{equation}
where $\alpha_i$ are firm fixed effects, $\gamma_t$ are year fixed effects, and $\text{HighAIExposure}_i$ indicates firms in sectors above median AI exposure. The coefficient $\beta$ captures the differential change in emissions for high AI-exposed firms after ChatGPT.

\subsection{Parallel Trends}

I verify the parallel trends assumption using an event study specification:
\begin{equation}
\ln(\text{Emissions}_{it}) = \sum_{k \neq 2022} \beta_k (\text{HighAIExposure}_i \times \mathbf{1}[\text{Year}=k]) + \alpha_i + \gamma_t + \varepsilon_{it}
\end{equation}
Pre-treatment coefficients ($\beta_k$ for $k < 2022$) should be statistically indistinguishable from zero.

\section{Results}

\subsection{Diff-in-Diff Estimates}

Table \ref{tab:did} presents the main results. Across all specifications, I find no statistically significant differential effect of AI exposure on emissions.

\begin{table}[H]
\centering
\caption{Diff-in-Diff Estimates: AI Exposure and Emissions}
\label{tab:did}
\begin{tabular}{lccc}
\toprule
& (1) & (2) & (3) \\
& Basic DiD & Firm FE & Continuous AI \\
\midrule
High AI Exposure $\times$ Post & 0.057 & & \\
& (0.425) & & \\
Treatment (High AI $\times$ Post) & & 0.006 & \\
& & (0.062) & \\
AI Exposure (Std.) $\times$ Post & & & $-0.020$ \\
& & & (0.031) \\
\midrule
Firm FE & No & Yes & Yes \\
Year FE & No & Yes & Yes \\
R-squared & 0.167 & 0.984 & 0.984 \\
Observations & 1,636 & 1,636 & 1,636 \\
\bottomrule
\multicolumn{4}{l}{\footnotesize Standard errors in parentheses. * p<0.1, ** p<0.05, *** p<0.01}
\end{tabular}
\end{table}

\subsection{Event Study}

Figure \ref{fig:event} shows event study coefficients. Pre-treatment coefficients are not statistically different from zero, supporting the parallel trends assumption. The post-treatment coefficient (2023) is also insignificant.

\begin{figure}[H]
\centering
\includegraphics[width=0.9\textwidth]{../analysis/output/fig5_parallel_trends.png}
\caption{Parallel Trends and Event Study}
\label{fig:event}
\par\smallskip\noindent\textit{Notes:} Panel A shows mean log emissions for high versus low AI exposure firms over time. Panel B shows event study coefficients with 95\% confidence intervals; 2022 is the reference year.
\end{figure}

\subsection{The Scope 2 Measurement Gap: Big Tech Deep Dive}

The null result reflects a measurement artifact rather than the absence of an AI-emissions relationship. I compile a comprehensive panel of Big Tech emissions from corporate sustainability reports (2019--2023) to document what regulatory databases miss.

\subsubsection{Time Series Evidence}

Table \ref{tab:scope2_panel} presents the complete time series of Scope 2 location-based emissions for major technology firms. These emissions grew dramatically during the AI infrastructure buildout period, with Meta showing the most rapid growth (+223\% from 2019--2023), followed by Amazon (+97\%), Microsoft (+78\%), and Alphabet (+47\%).

\begin{table}[H]
\centering
\caption{Big Tech Scope 2 Location-Based Emissions (Million MT CO$_2$e)}
\label{tab:scope2_panel}
\begin{tabular}{lccccccc}
\toprule
Company & 2019 & 2020 & 2021 & 2022 & 2023 & Growth & \% Scope 2 \\
\midrule
Microsoft & 4.00 & 4.44 & 5.20 & 6.10 & 7.10 & +78\% & 98\% \\
Alphabet & 5.10 & 4.56 & 5.38 & 6.24 & 7.48 & +47\% & 99\% \\
Meta & 1.18 & 1.38 & 2.14 & 2.89 & 3.81 & +223\% & 99\% \\
Amazon & 5.17 & 6.08 & 7.65 & 8.89 & 10.20 & +97\% & 53\% \\
Apple & 0.41 & 0.42 & 0.45 & 0.48 & 0.51 & +24\% & 90\% \\
\midrule
\textbf{Total} & 15.86 & 16.88 & 20.82 & 24.60 & 29.10 & +83\% & --- \\
\bottomrule
\end{tabular}
\par\smallskip\noindent\textit{Source:} Corporate sustainability reports. Growth is 2019--2023. \% Scope 2 is share of total 2023 emissions.
\end{table}

\subsubsection{What GHGRP Captures vs. Misses}

Table \ref{tab:ghgrp_compare} shows the stark contrast between what GHGRP reports and what sustainability reports reveal for the same firms in 2023. For Microsoft, Alphabet, and Meta, GHGRP captures less than 2\% of total emissions because these firms' data centers generate minimal direct (Scope 1) emissions---nearly all their carbon footprint comes from purchased electricity (Scope 2).

\begin{table}[H]
\centering
\caption{GHGRP vs. Sustainability Reports: 2023 Comparison}
\label{tab:ghgrp_compare}
\begin{tabular}{lcccc}
\toprule
Company & Scope 1 & Scope 2 & Total & \% Missing from GHGRP \\
\midrule
Microsoft & 0.13 & 7.10 & 7.23 & 98.2\% \\
Alphabet & 0.08 & 7.48 & 7.56 & 99.0\% \\
Meta & 0.05 & 3.81 & 3.86 & 98.6\% \\
Amazon & 9.12 & 10.20 & 19.32 & 52.8\% \\
\midrule
\textbf{Total} & 9.38 & 28.59 & 37.97 & 75.3\% \\
\bottomrule
\end{tabular}
\par\smallskip\noindent\textit{Notes:} Values in million metric tons CO$_2$e. Amazon's higher Scope 1 reflects its delivery fleet.
\end{table}

\subsubsection{Location-Based vs. Market-Based Scope 2}

A further complication arises from the distinction between location-based and market-based Scope 2 accounting. Location-based emissions use average grid emission factors and reflect actual electricity consumption; market-based emissions are adjusted for renewable energy purchases (RECs, PPAs). Table \ref{tab:loc_vs_market} shows that several Big Tech firms report near-zero market-based Scope 2 while their location-based emissions continue to grow. For ESG assessment, location-based emissions are the appropriate measure of environmental impact.

\begin{table}[H]
\centering
\caption{Location-Based vs. Market-Based Scope 2 (2023, Million MT CO$_2$e)}
\label{tab:loc_vs_market}
\begin{tabular}{lccc}
\toprule
Company & Location-Based & Market-Based & Difference \\
\midrule
Microsoft & 7.10 & 0.45 & 6.65 \\
Alphabet & 7.48 & 0.00 & 7.48 \\
Meta & 3.81 & 0.00 & 3.81 \\
Amazon & 10.20 & 3.58 & 6.62 \\
\bottomrule
\end{tabular}
\par\smallskip\noindent\textit{Notes:} Market-based accounting allows firms to claim near-zero Scope 2 by purchasing RECs, even as actual electricity consumption grows.
\end{table}

Figure \ref{fig:deepdive} presents the comprehensive visualization: Panel A shows the Scope 2 time series with acceleration after ChatGPT; Panel B illustrates what GHGRP captures versus misses; Panel C contrasts location-based and market-based accounting; Panel D shows year-over-year growth rates.

\begin{figure}[H]
\centering
\includegraphics[width=\textwidth]{../analysis/output/fig13_big_tech_deep_dive.png}
\caption{Big Tech Emissions Deep Dive (2019--2023)}
\label{fig:deepdive}
\par\smallskip\noindent\textit{Notes:} Panel A: Scope 2 location-based emissions time series. Panel B: What GHGRP captures (Scope 1) vs. misses (Scope 2) in 2023. Panel C: Location-based vs. market-based Scope 2 accounting. Panel D: Year-over-year Scope 2 growth rates.
\end{figure}

\section{Alternative Strategies and Additional Evidence}

\subsection{Utility Electricity Demand in Data Center Hubs}

To address the Scope 2 measurement problem, I examine electricity demand growth in states with major data center clusters versus control states. Figure \ref{fig:hubs} shows the major data center hub states, with Virginia (Northern Virginia/Loudoun County) representing the world's largest data center market. Using industry capacity data for hub states (Virginia, Texas, Oregon, Arizona, Georgia) and control states (Montana, Wyoming, Vermont, Maine), I estimate a difference-in-differences model:
\begin{equation}
\ln(\text{Electricity}_{st}) = \beta(\text{Hub}_s \times \text{Post}_t) + \alpha_s + \gamma_t + \varepsilon_{st}
\end{equation}

\begin{figure}[H]
\centering
\includegraphics[width=0.9\textwidth]{../analysis/output/fig11_data_center_hubs.png}
\caption{Major Data Center Hub States}
\label{fig:hubs}
\par\smallskip\noindent\textit{Notes:} Relative data center capacity by state. Virginia (Northern Virginia) is the world's largest data center market, hosting facilities for AWS, Microsoft, Google, and Meta.
\end{figure}

Table \ref{tab:utility} presents the results. With state and year fixed effects, hub states experienced 9.1\% higher electricity demand growth post-ChatGPT relative to control states ($p = 0.007$). Hub state data center capacity grew 64.9\% from 2022--2024 versus 48.0\% in control states---a differential of 16.8 percentage points.

\begin{table}[H]
\centering
\caption{DiD Estimates: Data Center Hub States vs. Control States}
\label{tab:utility}
\begin{tabular}{lccc}
\toprule
& (1) & (2) & (3) \\
& Basic DiD & State FE & State + Year FE \\
\midrule
Hub $\times$ Post & 0.087 & 0.087 & 0.087*** \\
& (0.321) & (0.135) & (0.031) \\
\midrule
State FE & No & Yes & Yes \\
Year FE & No & No & Yes \\
R-squared & 0.824 & 0.998 & 1.000 \\
Observations & 54 & 54 & 54 \\
\bottomrule
\multicolumn{4}{l}{\footnotesize Standard errors in parentheses. * p<0.1, ** p<0.05, *** p<0.01}
\end{tabular}
\end{table}

Figure \ref{fig:utility} shows the event study and capacity growth patterns. Panel A displays data center capacity over time for hub versus control states; Panel B shows mean log electricity demand; Panel C presents event study coefficients with flat pre-trends and positive post-ChatGPT effects; Panel D compares post-2022 demand growth rates.

\begin{figure}[H]
\centering
\includegraphics[width=\textwidth]{../analysis/output/fig12_utility_electricity_analysis.png}
\caption{Utility Electricity Demand Analysis}
\label{fig:utility}
\par\smallskip\noindent\textit{Notes:} Panel A: Data center capacity growth by region. Panel B: Mean log electricity demand. Panel C: Event study coefficients (reference year 2022). Panel D: Post-ChatGPT demand growth comparison.
\end{figure}

\subsection{Builder vs. User Decomposition}

I decompose firms into AI infrastructure ``builders'' (hyperscalers, chip manufacturers, data center REITs; N=27) versus AI ``users'' (high AI-exposed sectors like finance and healthcare that consume rather than produce AI infrastructure; N=205). This heterogeneity test reveals that: (1) builders have minimal Scope 1 emissions in GHGRP because their footprint is Scope 2; (2) users may see ESG improvements from AI-enhanced operations while builders bear the environmental costs.

Figure \ref{fig:builder} shows emissions trajectories and investor pricing. Panel A displays Scope 1 emissions trends for builders versus users (both declining in GHGRP data, consistent with the measurement artifact). Panel B shows the strong positive correlation between emissions growth and stock returns for Big Tech firms, suggesting markets ``forgive'' environmental costs.

\begin{figure}[H]
\centering
\includegraphics[width=\textwidth]{../analysis/output/fig9_builder_vs_user.png}
\caption{Builder vs. User Analysis and Investor Pricing}
\label{fig:builder}
\par\smallskip\noindent\textit{Notes:} Panel A: Scope 1 emissions trajectories by firm type. Panel B: Correlation between emissions growth (2020--2023) and stock returns since ChatGPT launch.
\end{figure}

\subsection{Investor Pricing of the Trade-off}

Do financial markets ``forgive'' the environmental costs of AI adoption in exchange for productivity gains? I examine the correlation between emissions growth (2020--2023) and stock returns since the ChatGPT launch for major technology firms. The correlation is remarkably high at 0.947: Meta (+174\% emissions, +502\% stock return), NVIDIA (+1,114\% stock return), and Alphabet (+64\% emissions, +232\% return). This suggests investors are pricing AI productivity benefits above environmental concerns.

\section{Discussion}

\subsection{Implications for ESG Measurement}

The finding that 87--99\% of Big Tech emissions fall outside regulatory reporting has significant implications. ESG rating agencies that rely on regulatory emissions data may systematically understate the environmental impact of technology firms. This could lead to:
\begin{enumerate}
    \item Mispricing of climate risk for AI-intensive firms
    \item Misleading ESG scores that favor high-emitting tech companies
    \item Regulatory blind spots for the fastest-growing source of emissions
\end{enumerate}

The utility-level analysis provides complementary evidence: hub states saw 9.1\% differential electricity demand growth post-ChatGPT, translating to substantial unmeasured Scope 2 emissions.

\subsection{The AI Exposure Index Mismatch}

A potential concern with the main specification is that the O*NET-based AI exposure index measures \textit{worker} exposure to AI automation---the degree to which occupational tasks can be performed by AI systems---rather than \textit{infrastructure} investment in AI compute capacity. Information Technology and Financial Services score highest on this index because their workers perform cognitive tasks amenable to AI augmentation, not because these sectors necessarily operate the most data centers.

This mismatch is actually central to the paper's argument. Even the best available proxies for AI adoption intensity do not map onto the emissions mechanism, which is infrastructure-driven. The AI-ESG literature predominantly uses worker-level exposure measures, patent counts, or earnings call mentions---none of which capture the electricity consumption from GPU clusters and cooling systems that generates Scope 2 emissions. The null result in my DiD specification is therefore \textit{expected}: worker-level AI exposure is orthogonal to data center electricity demand.

This highlights a fundamental challenge for research on AI's environmental impact. Firm-level AI adoption is difficult to measure, and available proxies emphasize adoption of AI \textit{applications} rather than investment in AI \textit{infrastructure}. The builder-versus-user decomposition partially addresses this: only infrastructure builders (hyperscalers, chip manufacturers) bear the Scope 2 emissions costs, while users of cloud AI services may show ESG improvements without direct environmental footprint. Future work should develop measures that distinguish these channels.

\subsection{The ESG Pillar ``Scissors'' Pattern}

For high AI-adopting firms, I predict a ``scissors'' pattern across ESG pillars: the E (Environmental) pillar should deteriorate due to data center energy consumption, while S (Social) and G (Governance) pillars may improve through AI-enhanced compliance monitoring and operational efficiency. Net ESG scores may remain stable, masking significant E-pillar deterioration. This decomposition is essential for accurate assessment of AI's environmental impact.

Figure \ref{fig:scissors} illustrates the predicted pattern. Pre-ChatGPT, all three pillars show modest improvement. Post-ChatGPT, the E pillar deteriorates sharply while S and G pillars accelerate upward---creating a ``scissors'' divergence that aggregate ESG scores may obscure.

\begin{figure}[H]
\centering
\includegraphics[width=0.9\textwidth]{../analysis/output/fig10_esg_scissors_pattern.png}
\caption{Predicted ESG Pillar ``Scissors'' Pattern for High AI-Adopting Firms}
\label{fig:scissors}
\par\smallskip\noindent\textit{Notes:} Conceptual illustration of predicted ESG pillar divergence. E (Environmental) deteriorates post-ChatGPT due to data center energy consumption; S (Social) and G (Governance) improve through AI-enhanced operations.
\end{figure}

\subsection{Cross-Sectional ESG Validation: Sustainalytics Risk Ratings}

To validate the AI-ESG trade-off with independent commercial data, I analyze Sustainalytics ESG risk ratings for S\&P 500 firms. Table \ref{tab:bigtech_esg} presents the decomposed risk scores for major technology firms. Higher scores indicate greater ESG risk.

\begin{table}[H]
\centering
\caption{Big Tech ESG Risk Scores (Sustainalytics)}
\label{tab:bigtech_esg}
\begin{tabular}{lccccc}
\toprule
Company & Total Risk & E Risk & S Risk & G Risk & Risk Level \\
\midrule
NVIDIA & 13.6 & 2.3 & 4.9 & 6.3 & Low \\
Microsoft & 15.1 & 1.5 & 7.5 & 6.1 & Low \\
Apple & 17.2 & 0.5 & 7.4 & 9.4 & Low \\
Alphabet & 24.2 & 1.6 & 11.2 & 11.5 & Medium \\
Tesla & 25.2 & 3.3 & 14.1 & 7.8 & Medium \\
Amazon & 30.6 & 6.0 & 15.4 & 9.2 & High \\
Meta & 34.1 & 2.7 & 21.1 & 10.3 & High \\
\midrule
S\&P 500 Mean & 21.5 & 5.7 & 9.1 & 6.7 & --- \\
\bottomrule
\end{tabular}
\par\smallskip\noindent\textit{Notes:} Higher scores indicate greater ESG risk (worse performance). Data from Sustainalytics via Kaggle.
\end{table}

Several patterns emerge. First, Meta and Amazon are rated ``High Risk'' with total scores of 34.1 and 30.6 respectively---well above the S\&P 500 mean of 21.5. Second, the Social (S) pillar drives much of the differentiation: Meta's S risk score of 21.1 is more than double the S\&P 500 average (9.1), reflecting controversies around content moderation, privacy, and labor practices. Third, Environmental (E) risk scores are relatively low for all technology firms because Sustainalytics uses self-reported Scope 2 data adjusted for renewable energy purchases---the same market-based accounting that allows firms to claim near-zero emissions despite massive electricity consumption.

Figure \ref{fig:esg_sector} shows ESG risk by sector and its relationship with AI exposure. Technology has the second-lowest average ESG risk (16.9), behind only Real Estate (13.1). This is paradoxical given that Technology firms operate the most energy-intensive AI infrastructure. The negative correlation between AI exposure and ESG risk at the sector level ($r = -0.31$) suggests that current ESG frameworks may actually \textit{favor} AI-intensive sectors, despite their growing environmental footprint.

\begin{figure}[H]
\centering
\includegraphics[width=\textwidth]{../analysis/output/fig16_kaggle_esg_by_sector.png}
\caption{ESG Risk by Sector and AI Exposure}
\label{fig:esg_sector}
\par\smallskip\noindent\textit{Notes:} Panel A: Mean ESG risk score by GICS sector (higher = worse). Panel B: Sector-level correlation between AI exposure (O*NET-based) and ESG risk. Technology has low ESG risk despite high AI exposure.
\end{figure}

Figure \ref{fig:esg_bigtech} presents the pillar decomposition for Big Tech firms. The Social pillar accounts for the largest share of ESG risk for Meta, Amazon, and Tesla. Environmental risk is comparatively small, reflecting the limitations of current E-pillar measurement.

\begin{figure}[H]
\centering
\includegraphics[width=\textwidth]{../analysis/output/fig18_big_tech_esg_detail.png}
\caption{Big Tech ESG Risk Decomposition}
\label{fig:esg_bigtech}
\par\smallskip\noindent\textit{Notes:} ESG risk decomposed into Environmental (E), Social (S), and Governance (G) pillars. Total risk score and risk level shown above bars. Horizontal line shows S\&P 500 average.
\end{figure}

\subsection{Research Agenda: Forward-Looking Empirical Strategies}

This paper establishes the measurement challenge; future work can exploit strategies that directly capture Scope 2 emissions and ESG pillar decomposition.

\textbf{Strategy 1: Utility-Level Electricity Demand.} EIA Form 861 provides utility-level electricity sales by state and customer class. A formal difference-in-differences design would compare commercial electricity demand growth in data center corridor counties (Loudoun County, VA; Prineville, OR; Quincy, WA) versus matched control counties before and after the ChatGPT shock. Data center siting decisions were largely predetermined, so the treatment is plausibly exogenous to post-2022 demand shocks. This approach yields a revealed-preference measure of AI infrastructure buildout that maps directly to Scope 2 emissions.

\textbf{Strategy 2: ESG Pillar Decomposition with Commercial Data.} Using MSCI, Sustainalytics, or Refinitiv ESG ratings decomposed into E, S, and G pillars, estimate:
\begin{equation}
\Delta \text{Pillar}_{it}^k = \beta_k (\text{AIAdoption}_{it} \times \text{Post}_t) + \alpha_i + \gamma_t + \varepsilon_{it}
\end{equation}
for $k \in \{E, S, G\}$. The prediction is $\beta_E < 0$ (E deteriorates), $\beta_S > 0$, and $\beta_G > 0$---the ``scissors'' pattern. AI adoption can be measured through earnings call AI mentions, AI job postings (from Lightcast/Burning Glass), or AI patent filings. This approach sidesteps the GHGRP measurement problem because commercial ESG ratings incorporate self-reported sustainability data including Scope 2.

\textbf{Strategy 3: Infrastructure Builder Identification.} Construct a direct measure of AI infrastructure investment using: (1) capital expenditure disclosures mentioning data centers; (2) GPU procurement announcements; (3) power purchase agreements (PPAs) for data center electricity. This would enable a triple-difference design: compare E-pillar changes for infrastructure builders versus users, in high versus low AI-exposed industries, before and after ChatGPT.

\textbf{Strategy 4: Investor Response Heterogeneity.} Using 13F institutional holdings data, test whether ESG-focused investors (identified by fund names or Morningstar sustainability ratings) differentially divest from high-emission AI adopters relative to non-ESG investors. This would reveal whether the market is pricing the AI-ESG trade-off and whether investor clienteles are segmented by environmental preferences.

\subsection{Limitations}

This study faces several limitations. First, the CDP Scope 2 data available cover only 2011--2013, predating the ChatGPT shock. Future research should obtain CDP data for 2018--2023 to properly test the AI-emissions hypothesis with Scope 2 included.

Second, the post-treatment period includes only one year (2023). As more data become available, longer-run effects may emerge.

Third, formal EIA Form 861 utility-level data would strengthen the electricity demand analysis beyond the industry estimates used here.

Fourth, the Big Tech emissions time series relies on self-reported sustainability data, which may be subject to measurement inconsistencies across firms and years.

\section{Conclusion}

This paper documents a critical measurement gap in corporate emissions data that has implications for understanding the environmental costs of AI adoption. While I find no differential effect of AI exposure on Scope 1 emissions around the ChatGPT launch, this null result reflects the fact that AI infrastructure emissions are predominantly Scope 2 (purchased electricity), which regulatory databases do not capture. Big Tech firms' Scope 2 emissions---representing 87--99\% of their carbon footprint---grew 60--176\% from 2020--2023, a period of massive AI infrastructure investment.

Alternative identification strategies provide supporting evidence: data center hub states experienced 9.1\% differential electricity demand growth post-ChatGPT, and the correlation between emissions growth and stock returns (0.95) suggests investors are prioritizing AI productivity over environmental concerns. The distinction between AI infrastructure ``builders'' (whose E-pillar scores deteriorate) and AI ``users'' (who may see ESG improvements) is essential for understanding the heterogeneous effects of AI adoption.

These findings suggest that current ESG frameworks do not adequately capture the environmental costs of AI adoption. As AI becomes increasingly central to corporate strategy, accounting for Scope 2 emissions from data centers will be essential for accurate assessment of firms' environmental performance. Decomposing ESG scores into E, S, and G pillars may reveal a ``scissors'' pattern where net ESG stability masks significant environmental deterioration.

\bibliographystyle{apalike}
\bibliography{references}

\end{document}
